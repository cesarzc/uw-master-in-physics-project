%
% Chapter 2
%
\chapter {Basics of Quantum Computing}

\section{Qubits}

In classical computation and classical information, a bit is the fundamental building block.
Analogously, in quantum computation and quantum information, a quantum bit or qubit is the fundamental building block.

Mathematically, a qubit is a linear combination of states $\ket{\psi}=\alpha\ket{0}+\beta\ket{1}$ where $\alpha$ and $\beta$ are complex numbers known as amplitudes and $\ket{0}$ and $\ket{1}$ are the computational basis states.

When a qubit is measured, the result is either $\ket{0}$ with probability $|\alpha|^2$ or $\ket{1}$ with probability $|\beta|^2$.
Since the probabilities must sum to one, the qubit's state is normalized:
$$|\alpha|^2+|\beta|^2=1$$

The computational basis states form an orthonormal basis represented by the following vectors:
$$\ket{0}=\left[\begin{array}{ccc}1 \\ 0\end{array}\right]$$
$$\ket{1}=\left[\begin{array}{ccc}0 \\ 1\end{array}\right]$$

Using the previous definitions, we can see a qubit as a unit vector in two-dimensional complex vector space:
\[\ket{\psi}=\left[\begin{array}{ccc}\alpha \\ \beta\end{array}\right]\]

\section{Quantum Gates}

\todo{Include the following:}

\begin{itemize}
    \item Single qubit gates
    \item Muti qubit gates
    \item Common gates
\end{itemize}

\section{Entanglement}

\section{Measurement}
