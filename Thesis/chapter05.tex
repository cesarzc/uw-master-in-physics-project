%
% Chapter 5
%
\chapter {Algorithm Execution Analysis Framework}

\section{Framework}

\todo{This was originally placed in the Introduction.}

\todo{Expand on the following structure of the thesis:}

\begin{enumerate}
    \item Implement an algorithm using fault-tolerant error-corrected gates.
    \item For each hardware platform do the following:
    \begin{enumerate}
        \item Use an open-system simulator to analyze and optimize the effectiveness of fault-tolerant error-corrected gates.
        \item Use a resource estimator to determine the maximum input size for the algorithm to run on a real NISQ device.
        \item Use a resource estimator to determine the characteristics that a device should have to run the algorithm for a specific input size, and determine what would be the runtime.
    \end{enumerate}
    \item Compare the results for each hardware platform.
\end{enumerate}

\todo{Briefly describe each step in the process}

\section{Open-System Simulator}

\todo{Describe what is the purpose of using an open-system simulator and how it works.}

\section{Resources Estimation Strategy}

The strategy we use for resources estimation is very similar to the one proposed by Soeken et al.\cite{ResourceEstimationFramework_Soeken_2021}. The process is the following:
\begin{enumerate}
    \item Implement a quantum algorithm using a high-level programming language (Q\# in this case).
    \item Verify the correctness of the implementation by executing the algorithm in a full state simulator.
    \item Setup the simulator to estimate physical resources with the parameters that are specific to a hardware platform.
    \item Analyze the results obtained from the resources estimator.
\end{enumerate}

\todo{Explain the following}
\begin{itemize}
    \item Resources Metrics: Describe the values obtained from the resources estimator (gate count, runtime, accumulated error), and how they are calculated.
    \item Gate Decomposition: Describe why logical-level gates have to be decomposed into physical-level gates.
    \item Limitations: Describe the limitations that this resources estimation has in regards to runtime (sum of the runtimes of individual gates rather than the critical path), and types of computations (trouble with mixed states).
\end{itemize}

\section{Extending Q\# Simulation Infrastructure for Estimation of Physical Resources}

Microsoft's Quantum Development Kit (QDK) supports the implementation of custom simulators that can be used to run Q\# programs. We leverage this capability and implement a simulator that calculates the resources a quantum algorithm would require to be executed in a hardware platform with specific characteristics.

\todo{Explain the following}
\begin{itemize}
    \item QDK Custom Simulators: Describe how custom simulators are implemented using diagrams and code snippets.
    \item Software Architecture of Physical Resources Estimator Simulator: Describe the software architecture using diagrams and code snippets.
\end{itemize}

Source code of a working version can be found in in \href{https://github.com/cesarzc/uw-master-in-physics-project}{GitHub}.
