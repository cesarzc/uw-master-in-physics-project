%
% Chapter 6
%
\chapter {Trapped-Ions Hardware Platform}

\todo{Briefly describe this quantum computing platform.}

\section{Native Gates and Platform Characteristics}

\todo{Enumerate the native gates that this platform implements, its characteristics (fidelity, gate time), and how logical gates are implemented (using circuits to illustrate them).}

\section{Fault-tolerant analysis and optimization}

\todo{Show the use of an open-system simulator to analyze and optimize the effectiveness of fault-tolerant error-corrected gates.}

\section{Resources Estimation Analysis}

\todo{Show (using tables and/or plots) how resources escalate as the size and pattern of the input changes}.

Example of ouput of resources used by the Bernstein-Vazirani algorithm using a secret string of size 5:
\begin{lstlisting}
    Ion Platform Resource Estimation
    Bernstein-Vazirani
    PHYSICAL LAYER
    Total Statistics
    ----------------
    Qubits: 6
    Gate Count: 38
    Time: 1175
    Error: 0.3500000000000003
    
    Gate Statistics
    ---------------
    R:
     - Count: 35
     - Time: 470
     - Error: 0.2300000000000002
    
    XX:
     - Count: 3
     - Time: 705
     - Error: 0.1200000000000001
\end{lstlisting}

\section{Analysis of Execution in NISQ Devices}

\todo{Analyze what would be the maximum size (and difficulty of pattern) of the secret string that can be used in a current NISQ device based on this platform.}

\section{Analysis of Execution of Algorithm for Input of Specific Size}

\todo{Use a resource estimator to determine the characteristics that a device should have to run the algorithm for a specific input size, and determine what would be the runtime.}
