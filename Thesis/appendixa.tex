%
% Appendix A
%
\chapter{Bernstein-Vazirani Q\# Implementation}
 
The following code presents a simple implementation of the Bernstein-Vazirani algorithm in the Q\# programming language:

\begin{lstlisting}
    namespace Quantum.BernsteinVazirani {

        open Microsoft.Quantum.Arrays;
        open Microsoft.Quantum.Canon;
        open Microsoft.Quantum.Intrinsic;
    
        function ArrayToString<'T> (array : 'T[]) : String
        {
            mutable first = true;
            mutable itemsString = "[";
            for item in array
            {
                if (first)
                {
                    set first = false;
                    set itemsString = itemsString + $"{item}";
                }
                else
                {
                    set itemsString = itemsString + $", {item}";
                }
            }
    
            set itemsString = itemsString + "]";
            return itemsString;
        }
    
        @EntryPoint()
        operation BernsteinVazirani () : Unit {
            Message("Bernstein-Vazirani");
            let secret = [One, Zero, One, One, Zero];
            use (qubits, aux) = (Qubit[Length(secret)], Qubit()) {
                X(aux);
                H(aux);
                ApplyToEach(H, qubits);
    
                // Oracle.
                for index in 0 .. Length(qubits) - 1 {
                    if (secret[index] == One){
                        CNOT(qubits[index], aux);
                    }
                }
    
                ApplyToEach(H, qubits);
                let results = ForEach(M, qubits);
                Message(ArrayToString<Result>(results));
                ResetAll(qubits);
                Reset(aux);
            }
        }
    }
\end{lstlisting}
