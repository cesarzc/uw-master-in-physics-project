%
% Chapter 4
%
\chapter {Shor's Semiprime Integer Factorization Algorithm}

Shor's algorithm is a polynomial-time quantum algorithm for semiprime integer factorization. In comparison, the time complexity of the most efficient known classical factoring algorithm is superpolynomial.

We chose Shor's algorithm to because it is one of the most significant quantum algorithms, and because the number of qubits and the number of quantum operations required are proportional to the input size. The number of operations is particullarly relevant since it makes the errors introduced by the physical gates an important consideration.

\section{Algorithm}

The problem that Shor's algorithm solves is the following: given a semiprime integer \textit{N}, find its two prime factor \textit{p} and \textit{q}.

\todo{Describe the algorithm.}

\section{Quantum Subroutine}

\todo{Show the circuit representation of the quantum subroutine.}

\section{Q\# Implementation}

The following Q\# code presents a top-level implementation of Shor's algorithm. It is a slight modification to the implementation found in Microsoft's quantum samples GitHub repository.

\todo{Add reference to Microsoft's quantum samples repository.}

\todo{Breakdown the implementation into different sections and describe each one (similar to what is done in the "Learn Quantum Computing with Python and Q\#" book).}

\todo{Consider creating a Q\# language option for lstlisting.}

\begin{lstlisting}[language=C]
@EntryPoint()
operation FactorSemiprimeInteger (N: Int) : (Int, Int) {
    // TODO: Paste implementation.
}
\end{lstlisting}

The following Q\# code presents a top-level implementation of the period-finding subroutine.

\begin{lstlisting}[language=C]
    @EntryPoint()
    operation EstimatePeriod(N : Int, a : Int) : Int {
        // TODO: Paste implementation.
        return 0;
    }
\end{lstlisting}
