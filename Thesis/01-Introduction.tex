%
% Chapter 1
%
\chapter {Introduction}

Algorithms designed for quantum computers have the potential to solve some problems that cannot be efficiently solved by algorithms designed for classical computers. However, estimating how much resources are needed to execute a quantum algorithm that outperforms a classical one is a difficult task. There are many quantum programming languages and tools built around them such as Q\#\cite{QSharp_Svore_2018}, Qiskit\cite{Qiskit_2021} and Cirq\cite{Cirq_2021} that allow execution of quantum algorithms on simulators but out-of-the-box options to estimate resources are limited to the logical level or not existent.

\section{The Purpose of This Thesis}

This thesis aims to estimate the resources required at the physical level to run Shor's semiprime integer factorization algorithm for trapped-ion and superconducting quantum hardware platforms. To do this, we will extend the simulators infrastructure built around Q\# to calculate the maximum number of physical qubits, the total number of physical gates, and the maximum runtime required to execute a particular quantum algorithm. 

Furthermore, we will also analyze the effects of errors introduced by the physical gates, analyze the feasibility of obtaining reliable results without implementing fault-tolerance, and estimate the cost of running the algorithm using error-corrected qubits and fault-tolerant gates.

In order to make this thesis more accesible to people from different backgrounds, we dedicate the next few chapters to provide a brief overview of the basics of quantum computing, quantum error correction, and Shor's semiprime integer factorization algorithm.
